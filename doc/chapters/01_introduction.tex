%!TEX root = ../thesis.tex
\chapter{Introduction}
\label{chap:Introduction}

\section{Emotions and Affect}
\label{sec:Emotions and Affect}

Within the context of this project it is important to distinguish between emotion and affect. Affect, in the context of this project will be treated as a term to associate predisposition towards stimuli. Thus, affect is in a sense, a general term that can be even used to describe animal, and other non-human entities. Emotions, on the other hand, are treated in this project as a state inherent to humans. This state is multidimensional, and every dimension, or emotion, can either be present in a certain amount, or not be present at all.
Emotions present an affect value, but not necessarily otherwise.

\subsection{Models of Emotions}
\label{sub:Models of Emotions}

When trying to detect emotion, it is relevant to know what emotions to look for. This is called a model of emotions, and is still a very discussed subject is psychology. Although there are several models of emotions the most persistent are Ekman’s model, Plutchikc’s model. The main assumption this work does follows the theory of constructed emotions, which recognizes affect as a physiological response to positive or negative stimuli, but emotions as a cognitive form of context-giving.

Why is it important to study emotion?
Emotions are considered a human state that influences behaviour and decision making. Many times, when expressing thoughts in a written or spoken form, one or several emotions are present. Detecting these emotions is an important task for human interaction. Automatic emotion detection on text is thus a machine learning task required for comprehensive human-computer interaction.

\subsubsection{Eckman's model of Emotions}
\label{subs:Eckman's model of Emotions}

What are emotions?
What is affect?
Why study emotions?
The role of emotions in communication
Measurements of emotions:
 - Facial Expressions
 - Biosignals
 - Language or self report
Constructed theory of emotions
Emotions in Language
Semantic Fields
Emotion Lexicon
Emotion Networks
Learning word semantics from context

\section{Emotion and Machine Learning}
\label{sec:Emotion and Machine Learning}
Language in Machine Learning
NLP workflow?
Language representation
Tokenization
The problem of large dictionaries
Dimensionality reduction/ autoencoders
Deep learning for language: automatic feature extraction
Machine Learning meaning from context
Word Embeddings
Transformers




\section{Problem setting}
\label{sec:Problem setting}


\section{Project Description}
\label{sec:Project Description}


\section{Objective}
\label{sec:Objective}


\section{Justification}
\label{sec:Justification}

Word Embeddings
Numerically representing words is commonly known as word embedding. This allows for Machine Learning (ML) models to easily manipulate text data that would otherwise be an arbitrary encoding of text. With the advances in machine learning automatic word embedding became a possible solution to avoid crowdsourcing. % This is too short. There is a lot of NLP using Machine Learning that doesn't use Word Embeddings.
Several machine learning approaches try to automatically learn the best numeric representation for characters, words, or sentences given the context of a dataset. Recently there have been many efforts from research institutions to generalize these embeddings though the use of powerful models, and bigger datasets. Such is the case of BERT, a model created by Google with massive datasets.

Affect learning and its relevance
The relevance of affect in text has increased since the popularization of text-based social networks, like twitter. There, individuals and organizations openly express their opinions. This creates an environment where implicit feedback about entities is present. An easy way to abstract popular opinion about a named entity is learning the affect expressed in text, such as a tweet. Affect can be a multidimensional phenomenon, but the most important dimension of it is valence: whether a text expresses positive or negative emotion.





\begin{figure}[H]
  \includegraphics[width=0.8\textwidth]{placeholder}
  \centering
  \caption{Example of an image}
  \label{fig:placeholder}
\end{figure}


In Figure~\ref{fig:placeholder} you can see an example of an image.


\begin{figure}[H]
  \begin{equation} \label{eq:equation_placeholder}
    \begin{split}
      A = \{&'a':['d','e','f','g'], \\
           &'b':['d','e','f','g'], \\
           &'c':['d','e','f','g'], \\
           &'d':['h'], \\
           &'e':['h'], \\
           &'f':['h'], \\
           &'g':['h'], \\
           &'h':[] \}
    \end{split}
  \end{equation}
  \caption{Example of an equation}
  \label{fig:equation_placeholder}
\end{figure}


\url{http://saifmohammad.com/WebPages/NRC-Emotion-Lexicon.htm}


\section{A Section}
\label{sec:A Section}


\subsection{A subsection}
\label{sub:A subsection}


\subsubsection{A sub-subsection}
\label{subs:A sub-subsection}
