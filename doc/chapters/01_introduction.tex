%!TEX root = ../thesis.tex
\chapter{Introduction}\label{chap:Introduction}

\section{Emotions and Affect}\label{sec:Emotions and Affect}

In the endless effort towards understanding human behaviour the phenomenon of Emotions has been recognized for centuries. It is after all, an experience that every human has. For many of us it represents a core variable in the representation of our biological, psychological, and social state. For such an important part of our lives, it is incredible how little we actually know about them. There is no scientific consensus on what an emotion is, and the term is often used to refer to mood, humor, temperament, personality, affect, character, and sentiment.
This section has as a goal to point out relevant discoveries and conceptualizations in the history of the study of emotions, as a way of delimiting the current study, and as a mean of introduction to the topic for technology-focused readers, but also as a way of outlining a working definition of Emotion, differenciating it from Affect.

% Hippocrates and Galen
Hippocrates, the father of modern medicine, defined the theory of the Four Humors.
This was based on the idea of humors, fluids, or chemicals that control human behaviour.
The four humors should be in balance within the body, and all diseases were caused by an inbalance of these.\cite{kalachanis2015hippocratic}
Galen at took this theory and created what can be considered as the first theory of personality. He believed that human bodies had a predisposition for unbalance of the humors. This made some people have a tendency to have more or less of these, and in turn, this would have an effect on their baseline behaviour. He described four different characteristical behaviours: phlegmatic, choleric, sanguine and melancholic.\cite{irwin1947galen}

% Darwin
Charles Darwin first described the importance of emotions in communication, and their relevance across cultures and even species. In his book \"The Expression of the Emotions in Man and Animals\" he writes \"...the young and the old of widely different races, both with man and animals, express the same state of mind by the same movements.\"~\cite{darwin1872emotions} He noticed that surprise was shown in humans across cultures, and even some mamals by raising the eyebrows. By framing emotions as a mean of communication, Darwin enabled the study of expression, and understanding of emotions as an evolutive advantage.

% Ekman
Although Emotions were thought to be universal there was no measurement of it. The universality of emotions was first formalized by Paul Ekman in his 1997 paper: "Universal facial expressions of emotion". Ekman studied facial anatomy, and expressions of different populations and cultures across the globe. He arrived to the conclussion that there are seven universal facial expressions of emotion:
Anger, Disgust, Fear, Surprise, Happiness, Sadness, and Contempt~\cite{ekman1997universal}\cite{ekman1992basic}.

% Picard
While emotions are a human concept, the digital advances of thte milenium caught up and integrated with the field, to create the Affective Computing. The concept was first coined by Rosalind Picard, who not only created it, but is also a lead researcher in the field~\cite{picard2000affective}. The concept was originaly related to Human-Computer interactions, and had the goal of computers expressing and recognizing emotions.

% Plutchik
Concerned with the subjectivity of emotion messurement, and the lack of generalization of self-report affect scales, Robert Plutchik proposed a method of meassuring the baseic emotions in a systematic way. He also proposed a way of derivating new emotions from the universal set proposed by Ekman. This was to be done based on theory, with enough diversity, but systematically relatable to the universal emotions~\cite{plutchik2013measurement}. In this way, the Plutchik model of emotions was created. This model is based on Ekman's universal emotions. Today, most Machine Learning tasks and datasets use this model of emotions. Different models will be discussed further in this chapter.

% Feldman
It is important to consider that for the last two decades, emotions have been studied based on Ekman's work on universal emotions. Although these do provide a framework for understanding how emotions came to be a part of the human experience, they do very little for their definition, or the description of emotions in language. Humans are complex, and even the most reliable model of universal emotions has exceptions. Psychologists interpret those differences with the help of the Theory of Constructed Emotions. This was first published by Lisa Feldman in 2014, and it describes the phenomenon of human emotions as a two-sided event: Affect and Emotion. Affect is a physiological phenomenon, the almost mechanical process that will enable behavioural response. The Emotional response is the cognitive contextualization of the former~\cite{feldman2014constructed}. Separation of physiological and cognitive responses allows the explanation of both, universality, and individual subjectivity.


% Definition of emotion


Within the context of this project it is important to distinguish between emotion and affect. Affect, in the context of this project will be treated as a term to associate predisposition towards stimuli. Thus, affect is in a sense, a general term that can be even used to describe animal, and other non-human entities. Emotions, on the other hand, are treated in this project as a state inherent to humans. This state is multidimensional, and every dimension, or emotion, can either be present in a certain amount, or not be present at all.
Emotions present an affect value, but not necessarily otherwise.


% Justification

When trying to detect emotion, it is relevant to know what emotions to look for. This is called a model of emotions, and is still a very discussed subject is psychology. Although there are several models of emotions the most persistent are Ekman’s model, Plutchikc’s model. The main assumption this work does follows the theory of constructed emotions, which recognizes affect as a physiological response to positive or negative stimuli, but emotions as a cognitive form of context-giving.

Why is it important to study emotion?
Emotions are considered a human state that influences behaviour and decision making. Many times, when expressing thoughts in a written or spoken form, one or several emotions are present. Detecting these emotions is an important task for human interaction. Automatic emotion detection on text is thus a machine learning task required for comprehensive human-computer interaction.

% Why does it make sense to study emotions in text?
% Why does emotion in tweets work?
% Why doesn't it work on transcript of conversations or friends?
% Why does emotional analysis need self-contained texts? Context

% Distinction from affect

% The role of emotions in communication

Measurements of emotions:
\begin{itemize}
  \item Facial Expressions
  \item Biosignals
  \item Language or self report
\end{itemize}

\subsection{Models of Emotions}\label{sub:Models of Emotions}

% Emotions in Language
% Delimit to language models of emotions

\subsubsection{Ekman's model of Emotions}\label{subs:Ekman's model of Emotions}

Semantic Fields
Emotion Lexicon
Emotion Networks
Learning word semantics from context


\section{Emotion and Machine Learning}\label{sec:Emotion and Machine Learning}
Language in Machine Learning
NLP workflow?
Language representation
Tokenization
The problem of large dictionaries
Dimensionality reduction/ autoencoders
Deep learning for language: automatic feature extraction
Machine Learning meaning from context
Word Embeddings
Transformers

\section{Problem setting}\label{sec:Problem setting}


\section{Project Description}\label{sec:Project Description}


\section{Objective}\label{sec:Objective}
%TODO:

Objetivo general:
Análizar la representación de emociones en modelos del lenguaje en machine learning. Esto de manera objetiva y medible, y subjectiva, pero disponible al público para fomentar discución sobre la representación de emociones en modelos de machine learning.

Objectivos secundarios
Plantear una metodología para análisar un dataset de emociones, basado en modelos de lenguaje previamente entrenados.

Cómo consecuencia del objetivo anterior, a marco teoríco y práctico se puede crear para el análisis de eficiencia de modelos previamente entrenados en datasets de clasificación de textos cortos, acompañados de una sola etiqueta. Ya que el resultado no está vinculado a la semántica de la etiqueta: (emociones). Pero la representatividad adecuada de las etiquetas en el espacio abstracto propuesto por el modelo, resulta en la más fácil clasificación de las observaciones y sus respectivas etiquetas.

\section{Justification}\label{sec:Justification}

Word Embeddings
Numerically representing words is commonly known as word embedding. This allows for Machine Learning (ML) models to easily manipulate text data that would otherwise be an arbitrary encoding of text. With the advances in machine learning automatic word embedding became a possible solution to avoid crowdsourcing. % TODO: {Wartena} This is too short. There is a lot of NLP using Machine Learning that doesn't use Word Embeddings.
Several machine learning approaches try to automatically learn the best numeric representation for characters, words, or sentences given the context of a dataset. Recently there have been many efforts from research institutions to generalize these embeddings though the use of powerful models, and bigger datasets. Such is the case of BERT, a model created by Google with massive datasets.

Affect learning and its relevance
The relevance of affect in text has increased since the popularization of text-based social networks, like twitter. There, individuals and organizations openly express their opinions. This creates an environment where implicit feedback about entities is present. An easy way to abstract popular opinion about a named entity is learning the affect expressed in text, such as a tweet. Affect can be a multidimensional phenomenon, but the most important dimension of it is valence: whether a text expresses positive or negative emotion.


% We're ultimately creating a method to make use of visualizations to encourage intuitions that can be developed into actual analysis methods to be used in the analysis of language embeddings.


\begin{figure}[H]
  \includegraphics[width=0.8\textwidth]{placeholder}
  \centering
  \caption{Example of an image}
\end{figure}\label{fig:placeholder}



Followed by equation~\ref{eq:equation_placeholder}
\begin{equation} \label{eq:equation_placeholder}
  \begin{split}
    A = \{&'a':['d','e','f','g'], \\
         &'b':['d','e','f','g'], \\
         &'c':['d','e','f','g'], \\
         &'d':['h'], \\
         &'e':['h'], \\
         &'f':['h'], \\
         &'g':['h'], \\
         &'h':[] \}
  \end{split}
\end{equation}
