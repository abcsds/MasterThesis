%!TEX root = ../thesis.tex
\chapter{Conclusion}\label{chap:Conclusion}

% Discussion
% Talk about the pedagogic value of this work

% Problems to be solved

% Future Work
% Talk about multi-label datasets, why they make sense, and why they should be looked into.
% How could they be looked into?

% New method for analysing the abstraction of concepts on pre-trained language models
Thanks to the work done in this project a new method for the creation of an emotional model can be created. A prospect model of emotions in text can be created by using explicit textual exrpessions of emotion. An example of an explicit expression of emotion is: \"I feel sad\" By removing the emotional word from this sentence, we can obtain an emotion-neutral emotion context. By parsing a corpus for these specific sentences, maping them on to a pre-trained-model vector space, and reproducing a separation method like PCA, the transformation that better captures the different emotional contexts can be obtained. This transformation can latter be used for non-explicit expressions of emotion.
This model of emotions can be context and languge specific, but the methodology is not restricted to any language, or even expressions of emotion. It is in general a method for analyzing conceptualizations of pre-trained language models.

% Ablative studies and their implications


% Talk about the horrors of accessing a dataset

% Talk about the crowdflower dataset and its weak points
% Talk about the EmotionX emotion tasks and its weak points
