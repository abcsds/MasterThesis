%!TEX root = ../thesis.tex
\chapter{Related Work}\label{chap:Related Work}

Mohamad and Turney created an Emotion Lexicon through crowdsourcing~\cite{mohammad_crowdsourcing_2013}. In this way an emotional word embedding was created by subjectively asking participants whether or not a word was related to a specific emotion.

Vo and Zhang created an automatic approach to learning sentiment lexicons for short texts through the use of emojis~\cite{vo_dont_2016}. This method uses the intrinsic usage of emojis to express positive or negative valence in a sentence, and exploded this to expand that valence to words used in the same context.

Maas et al\. created a method to learn word vectors for sentiment analysis~\cite{maas_learning_2011}.

By applying machine learned automatic embeddings, the creation of word embeddings based only on text data was open as a possibility. This is also a method that later became the popular Word2Vec method~\cite{mikolov_distributed_2013}.

A refining of word embeddings has been suggested by Yu et al\. by means of a clustering algorithm on the vector space~\cite{yu_refining_2017}.

Rothe et al\. suggested an orthogonal transformation to word embeddings used on SemEval2015 which yielded ultradense word embeddings for affect\cite{rothe_ultradense_2016}.

A further exploration of transformations of a word vector space was done by Hollis et al\. by means of component analysis, thus creating models of semantics from text. These were applied to affect\cite{hollis_principals_2016}.

These studies have mostly been done with affect: positive and negative valences, but have mostly ignored other emotional dimensions.


Talk about the models used and their history

Datasets
There is the possibility of using the datasets from the unified dataset of emotion in text. (An Analysis of Annotated Corpora forEmotion Classification in Text). This includes:
\begin{itemize}
  \item AffectiveText
  \item Blogs
  \item CrowdFlower
  \item DailyDialogs
  \item Electoral-Tweets
  \item EmoBank
  \item EmoInt
  \item Emotion-Stimulus
  \item fb-valence-arousal
  \item Grounded-Emotions
  \item ISEAR
  \item Tales
  \item SSEC
  \item TEC
\end{itemize}


The link to these datasets can be found under the github repository for the unified emotion datasets.
https://github.com/sarnthil/unify-emotion-datasets/tree/master/datasets

During the exploratory phase of this project, a subset of these datasets will be selected, giving priority to those with a more standardized, cleaner, or easier to access data.
Models
For word embeddings, different pre-trained models will be used:
\begin{itemize}
  \item Fasttext
  \item Word2Vec
  \item GloVe
  \item BERT
\end{itemize}

A word embedding can also be trained from data. This is considered an optional model to analyze.

The following techniques have been used:
\begin{itemize}
  \item TSNE
  \item PCA
\end{itemize}

Research Question

As a general research question we propose to answer the following:
When using pre-trained models for word and sentence embedding, \textbf{is the information about the emotional and affective content or context of the word or sentence represented in the vector space?}

This question can be approached in three different ways:
\begin{itemize}
  \item Is there a direct correlation between any of the dimensions of the vector space and human-labeled emotions and affect?
  \item Is there a linear transformation that will yield a direct correlation to the same human-labeled emotions?
  \item Is there a hierarchical structure that accurately represents the embedding of said labels?
\end{itemize}
